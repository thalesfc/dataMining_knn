\documentclass[a4paper, 11 pt, onecolumn]{article}   % list options between brackets
\usepackage{graphicx, url}              % list packages between braces

\usepackage{enumerate}


\usepackage{fancyvrb}
\usepackage{amssymb,amsmath}
%\usepackage[top=3cm, bottom=3cm, left=3cm, right=3cm]{geometry}

\usepackage[utf8]{inputenc}
\usepackage[brazilian]{babel} 

% type user-defined commands here
\newcommand{\HRule}{\rule{\linewidth}{0.5mm}}
\newcommand{\barra}{\\[0.5cm]}

\hyphenation{ne-ga-ti-va}

\begin{document}

\title{Mineração de Dados - tp 3\\ KNN}   % type title between braces
\author{Thales Filizola Costa\\ \scriptsize{\textit{thalesfc@dcc.ufmg.br}}}         % type author(s) between braces
\date{\today \\ \HRule}    % type date between braces
\maketitle

\section{Introdução}
A seção \ref{basededados} retrata a base de dados utilizada neste trabalho, tendo como foco a análise de stemming e de retirada de stopwords e relacionando os mesmos com os efeitos na acurácia e na performance do algoritmo. 

 
\section{Base de Dados}
\label{basededados}
A base de dados utilizada consiste em \textit{reviews} de usuários para filmes. Para cada \textit{review} são fornecidos o comentário e a classificação do comentário, que pode ser positiva (1) ou negativa (0).

\subsection{Vocabulário}

Nessa subseção queremos determinar se vamos utilizar ou não \textit{stemming} e se vamos retirar ou não \textit{stopwords} para os próximos experimentos. 

A tabela \ref{tab:string} apresenta as diferentes configurações que foram executadas. Nesses experimentos, o número de vizinhos utilizados foi dez (\texttt{k=10}) e a distância entre os pontos foi calculada utilizando a distância euclideana (\texttt{d=euclidean}).


%%  --------------- end of document
\nocite{meira}
\nocite{urlMetrics}

\bibliographystyle{plain}
\bibliography{reuniao}

\end{document}